%----------------------------------------------------------------------------------------
% Curriculum Vitae Main TeX Document
%----------------------------------------------------------------------------------------
% Article class because we want to fully customize the page and not use a cv template
\documentclass[a4paper, 11pt]{article}

% fontspec allows you to use TTF/OTF fonts directly
% \usepackage{fontspec}
% \defaultfontfeatures{Ligatures=TeX}
% \setmainfont[Path=fonts/, BoldFont=Outfit-Medium]{Outfit-Light}
\usepackage{mathspec}
\setallmainfonts(Digits,Latin)[Path=fonts/, BoldFont=Outfit-Medium]{Outfit-Light}

% Icons
\usepackage{fontawesome5}
\usepackage{academicons}

% Colours and graphics
\RequirePackage{color}
\RequirePackage{graphicx}
\usepackage[usenames, dvipsnames]{xcolor}
\definecolor{exeter_night_green}{HTML}{022020}
\definecolor{exeter_dark_green}{HTML}{003c3c}
\definecolor{exeter_deep_green}{HTML}{007d69}
\definecolor{exeter_bright_green}{HTML}{00c896}
\definecolor{exeter_highlight_green}{HTML}{00dca5}
\newcommand{\highlight}[1]{\textbf{\textcolor{exeter_bright_green}{#1}}}

% Geometry
\usepackage[scale=1.0]{geometry}
\geometry{a4paper, top=10mm, bottom=10mm, left=10mm, right=10mm, heightrounded, includehead, headheight=18pt}

% Headers
\usepackage{lastpage}
\usepackage{fancyhdr}
\pagestyle{fancy}
\fancyhead{}
\fancyfoot{}
\fancyhead[C]{\scriptsize\textcolor{exeter_deep_green}{Denis Sergeev's CV\\\thepage~/~\pageref{LastPage}}}
\renewcommand{\headrulewidth}{0pt}

% Paragraph formatting
\usepackage{parskip}

% Lists and tables
\usepackage[inline]{enumitem}
\usepackage{tabularx}
\usepackage{ltablex}
\usepackage{multicol}
\usepackage{multirow}

% Section formatting
\usepackage{titlesec}
\titleformat{\section}{\Large\bfseries\raggedright}{}{0em}{}[\titlerule]
\titlespacing{\section}{0pt}{10pt}{10pt}
% \titlespacing*{\section}{0pt}{-0.05\baselineskip}{0.75\baselineskip}
\titlespacing*{\subsection}{0pt}{-0.75\baselineskip}{0.25\baselineskip}

% Links
\usepackage{url}
\usepackage[unicode, draft=false]{hyperref}
\hypersetup{
colorlinks,
breaklinks,
urlcolor=exeter_deep_green,
linkcolor=exeter_deep_green,
pdfauthor={Denis Sergeev},
pdftitle={Denis Sergeev's CV}
}

% Date formatting
\usepackage[nodayofweek]{datetime}
\newdateformat{mydate}{\THEDAY\ \shortmonthname[\THEMONTH]\ \THEYEAR}

% Units formatting
\usepackage{siunitx}
\sisetup{group-separator={\,}}
\newcommand{\estval}[1]{$\sim$\pounds\num{#1}}

% To debug page outer frames
% \usepackage{showframe}

%----------------------------------------------------------------------------------------
\begin{document}
\thispagestyle{empty}

% Needed to keep the tabularx columns stretched over the whole linewidth when using ltablex
\keepXColumns

%----------------------------------------------------------------------------------------
% Header
%----------------------------------------------------------------------------------------
\noindent\vspace{-20mm}
\begin{tabularx}{\linewidth}{@{}p{0.275\linewidth} p{0.525\linewidth} X@{}}
\multicolumn{3}{@{}l@{}}{{\Huge\textbf{\textcolor{exeter_bright_green}{Denis Sergeev}}}} \\
& & \\
\href{https://pronouns.org/what-and-why}{\,\faUser~Pronouns: he/him/his} &
\multirow{5}{*}{%
\begin{minipage}[t]{\linewidth}
\includegraphics[width=\linewidth]{images/sergeev_ads_metrics.pdf}
\end{minipage}
} &
\multirow{5}{*}{%
\begin{minipage}[t]{\linewidth}
\begin{tabular}{@{}r l@{}}
\input{stats}
 & \\
\multicolumn{2}{r}{\tiny \href{https://ui.adsabs.harvard.edu/public-libraries/2IT653szTA-gYw3vyA-9Sw}{\aiADS\,Updated:\,\mydate\today}}
\end{tabular}
\end{minipage}
} \\
\href{https://www.bristol.ac.uk/physics/research/astrophysics}{\faHome~University of Bristol, UK} & &\\
\href{mailto:denis.sergeev@bristol.ac.uk}{\faEnvelope~denis.sergeev@bristol.ac.uk} & &\\
\href{http://orcid.org/0000-0001-8832-5288}{\aiOrcid~0000-0001-8832-5288} & &\\
\href{https://dennissergeev.github.io}{\faGlobe~dennissergeev.github.io} & &\\
\href{https://github.com/dennissergeev}{\faGithub~dennissergeev} & &\\
\end{tabularx}
\vspace{-5mm}


%----------------------------------------------------------------------------------------
% Summary
%----------------------------------------------------------------------------------------
\section{About}
I am fascinated by planetary atmospheres.
In my research, I try to unravel the complexity of atmospheric processes on terrestrial planets using an hierarchy of 3D climate models.
I am particularly interested in the role of convection and clouds in the planetary habitability.

%----------------------------------------------------------------------------------------
% Education and experience
%----------------------------------------------------------------------------------------
\section{Academic Career}
\begin{tabularx}{\linewidth}{@{}l X@{}}
2021--     & Postdoctoral Research Fellow at the Department of Physics \& Astronomy, \textbf{University of Exeter} \\
2018--2021 & Postdoctoral Research Fellow at the Department of Mathematics \& Statistics, \textbf{University of Exeter} \\
2014--2018 & PhD in Meteorology at the \textbf{University of East Anglia} \\
& Supervisors: Ian A. Renfrew, Thomas Spengler, Stephen Dorling \\
& Thesis title (shortened): \href{https://ueaeprints.uea.ac.uk/id/eprint/68204/}{``Characteristics of Polar Lows in the Nordic Seas''} \\ %  and the Impact of Orography and Sea Ice on Their Development
2009--2014 & Specialist Diploma in Meteorology (1\textsuperscript{st} class equiv.) at the \textbf{Lomonosov Moscow State University}  \\
& Supervisor: Victor Stepanenko \\
& Thesis title: \href{https://figshare.com/articles/thesis/SergeevDE_diploma_pdf/5326846/1}{``Idealised Numerical Modelling of Polar Mesocyclone Dynamics''}\\
% Oct 2013 & Visitor at the \textbf{Geophysical Institute, University of Bergen} \\
% & Supervisor: Prof. Thomas Spengler \\
% Jul 2012 & Intern at the \textbf{A.M. Obukhov Institute of Atmospheric Physics} \\
% & Supervisor: Dr. Alexey Eliseev \\
\end{tabularx}

%----------------------------------------------------------------------------------------
% Awards
%----------------------------------------------------------------------------------------
\section{Awards and Scholarships}
\begin{tabularx}{\linewidth}{@{}lXr@{}}
\multicolumn{2}{@{}l}{\highlight{Direct Funding, PI}} & \highlight{Est. Total Value} \\
2022 & Undergraduate Student Bursary (awarded; student declined) | RAS & \estval{1200} \\
2017 & Best Presentation Award | CEEDA Symposium & \estval{100} \\
2016 & Travel Bursary | Polar Prediction School & \estval{1000} \\
2015 & Travel Award | High-Latitude Dynamics workshop & \estval{1000} \\ % https://highlatdynamics.w.uib.no/
2014 & Lord Zuckerman PhD scholarship | School of Environmental Sciences, UEA & \estval{112000} \\
2014 & Young Scientist Travel Award | EGU General Assembly & \estval{200} \\
2014 & Russian Academy of Sciences Young Scientist Medal & \estval{1000} \\
\end{tabularx}

%----------------------------------------------------------------------------------------
% Publications
%----------------------------------------------------------------------------------------
\section{Publications}
{\scriptsize Citations\\(refereed in \textbf{bold})}
\begin{itemize}
\input{pubs}
\end{itemize}

%----------------------------------------------------------------------------------------
% Conferences
%----------------------------------------------------------------------------------------
\section{Conferences}
\subsection*{Invited Talks}
\begin{tabularx}{\linewidth}{@{}l X@{}}
Apr 2022 & \href{https://youtu.be/0uDBIp_EQrg}{Dichotomy of the atmospheric circulation on TRAPPIST-1e} \\ & NASA GISS Seminar | Online \\
Jan 2022 & Dichotomy of the atmospheric circulation on TRAPPIST-1e \\ & NASA GSFC Extrasolar Planets Seminar | Online \\
Nov 2021 & TRAPPIST-1 Habitable Atmosphere Intercomparison (THAI) \\ & MPIA APEx Exocoffee | Online \\
May 2021 & \href{https://youtu.be/ZUfIK-HMgLw}{Overcast on TRAPPIST-1e} \\ & RCC MSU Geophysical Seminar | Online \\
Sep 2020 & \href{https://youtu.be/E4UAoCoI1x8}{Simulations of convection over a range of atmospheric conditions on TRAPPIST-1e} \\ & THAI Workshop | Online \\
Apr 2020 & \href{https://slides.com/denissergeev/2020-04-27-uor-met}{Atmospheric convection plays a key role in the climate of tidally locked exoplanets} \\ & University of Reading Meteorology Seminar | Online \\
Apr 2020 & \href{https://slides.com/denissergeev/2020-04-21-nasa-giss}{Atmospheric convection plays a key role in the climate of tidally locked exoplanets} \\ & NASA GISS Seminar | Online \\
\end{tabularx}
\vspace{1ex}
\subsection*{Contributed Talks}
\begin{tabularx}{\linewidth}{@{}l X@{}}
Jul 2022 & Bistability of the atmospheric circulation on TRAPPIST-1e \\ & Rocky Worlds II | Oxford/Online \\
Apr 2022 & Dichotomy of the atmospheric circulation on TRAPPIST-1e \\ & Exoplanet modelling in the James Webb era II: Terrestrial planets and subneptunes | Online \\
Nov 2020 & \href{https://slides.com/denissergeev/2020-11-10-um-workshop-sergeev}{Explicit convection on tidally locked rocky exoplanets simulated with the UM nesting suite} \\ & Unified Model users workshop | Online \\
Aug 2019 & \href{https://youtu.be/9nGIpQiPwDs}{Simulations of moist convection on tidally-locked rocky exoplanets} \\ & Exoclimes V | Oxford, UK \\
Jun 2019 & \href{https://speakerdeck.com/dennissergeev/north-atlantic-polar-mesoscale-cyclones-in-era5-and-era-interim-reanalyses}{North Atlantic polar mesoscale cyclones in ERA5 and ERA-Interim reanalyses} \\ & IGP workshop | Norwich, UK \\
Apr 2019 & Atmospheric convection on tidally-locked Earth-like exoplanets \\ & UK Exoplanet Community Meeting | London, UK \\
Jun 2018 & \href{https://speakerdeck.com/dennissergeev/polar2018}{Modification of Polar Low Development by Sea Ice and Svalbard Orography} \\ & POLAR2018 | Davos, Switzerland \\
Oct 2017 & \href{https://figshare.com/articles/The_influence_of_Svalbard_orography_and_sea_ice_on_polar_low_development/5510416}{The influence of Svalbard orography and sea ice on polar low development} \\ & 18th Cyclone Workshop | Sainte-Adèle, Canada \\
Apr 2017 & \href{http://dennissergeev.github.io/ceeda2017}{Polar lows and how background environment can influence their development} \\ & Cambridge Earth Systems Science EnvEast Doctoral Alliance Symposium | Cambridge, UK \\
May 2016 & Structure of the shear-line polar low south of Svalbard \\ & NORPAN kick-off meeting | Tokyo, Japan \\
Apr 2016 & \href{http://slides.com/denissergeev/deck}{Structure of the shear-line polar low south of Svalbard} \\ & 13th European Polar Lows Working Group Workshop | Paris, France \\
\end{tabularx}
\vspace{1ex}
\subsection*{Poster Presentations}
\begin{tabularx}{\linewidth}{@{}l X@{}}
Nov 2022 & Dry Modern-Day Mars Climate in the Met Office Unified Model \\ & UK Solar System Planetary Atmospheres | London, UK \\
Sep 2022 & Bistability of the Atmospheric Circulation on TRAPPIST-1e \\ & UK Exoplanet Community Meeting | Edinburgh, UK \\
Jul 2015 & Structure and dynamics of a shear-line polar low during a cold-air outbreak over the Norwegian Sea \\ & Royal
Meteorological Society Student Conference | Birmingham, UK \\
Mar 2015 & Structure and dynamics of a shear-line polar low during a cold-air outbreak over the Norwegian Sea \\ & Dynamics of Atmosphere-Ice-Ocean Interactions in the High Latitudes workshop | Rosendal, Norway \\
May 2014 & Numerical modelling of polar mesocyclones dynamics diagnosed by the energy budget \\ & EGU General Assembly | Vienna, Austria \\
Apr 2013 & Impact of subgrid-scale vegetation heterogeneity on the carbon cycle \\ & EGU General Assembly | Vienna, Austria \\
Apr 2013 & Numerical modelling of polar mesocyclones generation mechanisms \\ & EGU General Assembly | Vienna, Austria \\
\end{tabularx}

%----------------------------------------------------------------------------------------
% Supervision
%----------------------------------------------------------------------------------------
\section{Supervision}
I am an integral member of the \href{https://exoclimatology.com}{Exeter Exoplanet Theory Group (EETG)}, and have been actively involved in the supervision of students --- both as a \highlight{lead supervisor} and as a co-supervisor. Undergraduate and Masters students who went on to do an MSc/PhD are marked with *.
\\
\subsection*{PhD Supervision (1)}
\begin{tabularx}{\linewidth}{@{}l X@{}}
Sep 2021--Sep 2025 & Martha (Mei Ting) Mak \\
                  & Project: Hazes in Planetary Atmospheres \\
                  & Co-supervisors: Prof. N. J. Mayne, Dr. E. Hébrard, Dr. J. Manners \\
\end{tabularx}

\subsection*{Masters Supervision (9)}
\begin{tabularx}{\linewidth}{@{}l X@{}}
Sep 2020--Sep 2022 & Danny McCulloch* (MSci by Research) \\
                  & Project: Climate Modelling of Modern-Day Mars \\
                  & Co-supervisors: Prof. N. J. Mayne, Prof. M. Bate \\
Apr 2021--Sep 2022 & Meghan Plumridge* (MSci by Research) \\
                  & Project: Climate Modelling of Early Mars \\
                  & Co-supervisors: Prof. N. J. Mayne, Prof. M. Bate \\
Jan 2021--May 2022 & Jasper Chadwick \& Esse Sellwood \\
                  & Project: Ocean Heat Transport on Rocky Exoplanets \\
                  & Co-supervisors: Prof. F. H. Lambert, J. Eager-Nash \\
Jan 2021--May 2022 & Isabelle Browne \& Oakley Young \\
                  & Project: Greenhouse Effect on Early Mars \\
                  & Co-supervisors: Prof. N. J. Mayne, Prof. F. H. Lambert, J. Eager-Nash \\
Jan 2020--May 2021 & Toby Ferrison \\
                  & Project: Titan Climate Modelling \\
                  & Co-supervisor: Prof. F. H. Lambert \\
Oct 2018--May 2019 & Jake Eager-Nash* \& David Reichelt \\
                  & Project: Implications of Stellar Type on the Climate of Tidally Locked Terrestrial Exoplanets \\
                  & Co-supervisors: Prof. N. J. Mayne, Prof. F. H. Lambert \\
\end{tabularx}
\subsection*{Undergraduate and Summer Internship Supervision (7)}
\begin{tabularx}{\linewidth}{@{}l X@{}}
Jul--Sep 2022 & Oakley Young \\
              & Project: Ekman Ocean Model \\
              & Co-supervisors: J. Eager-Nash, Prof. F. H. Lambert \\
\highlight{Jun--Sep 2022} & \highlight{James McDermott* \& Lottie Woods*} \\
                          & \highlight{Project: Simulations of Lightning Storms on Tidally Locked Rocky Exoplanets} \\
Jun--Aug 2021 & Oakley Young \\
              & Project: Climate Modelling of Archean Earth \\
              & Co-supervisors: J. Eager-Nash, Prof. N. J. Mayne \\
Jun--Aug 2021 & Joshua Parkin \& Esse Sellwood \\
              & Project: The Impact of Host Star Spectrum on the Climate of Rocky Exoplanets \\
              & Co-supervisors: J. Eager-Nash, Prof. N. J. Mayne \\
Jun--Aug 2019 & Isobel Parry* \\
              & Project: Water Cycle on Proxima Centauri b \\
              & Co-supervisor: Prof. F. H. Lambert \\
\end{tabularx}

%----------------------------------------------------------------------------------------
% Teaching
%----------------------------------------------------------------------------------------
\section{Teaching}
\begin{tabularx}{\linewidth}{@{}l X@{}}
Jan 2018 & \href{https://ueapy.github.io/pythoncourse2018}{ECR course ``Introduction to Python in Environmental Sciences''} \\ & Course creator \& leader | University of East Anglia \\
2015--2017 & Teaching assistance | Modelling Environmental Processes, Meteorology, Numerical Skills \\ & University of East Anglia \\
Apr 2017 & Field Course teaching assistance \\ & Slapton / University of East Anglia \\
Nov 2016 & \href{https://ueapy.github.io/enveast_python_course}{Python training course} \\ & Course creator \& leader | University of East Anglia \\
\end{tabularx}

%----------------------------------------------------------------------------------------
% Skills
%----------------------------------------------------------------------------------------
\section{Skills}
\begin{tabularx}{\linewidth}{@{}l X@{}}
Languages & English (fluent), French (basic), Russian (native speaker) \\
Numerical models &  Met Office Unified Model, LFRic, SOCRATES, LAGRANTO, Isca \\
Programming languages &  Python, FORTRAN, MATLAB, NCL \\
Python libraries & cartopy, cython, iris, matplotlib, numpy, pandas, pyvista, xarray \\
Parallel computing & Dask, MPI, OpenMP \\
Version control & Git, Subversion \\
Document preparation & \LaTeX, Markdown, HTML, CSS, reST
\end{tabularx}

%----------------------------------------------------------------------------------------
% Training
%----------------------------------------------------------------------------------------
\section{Vocational Training}
\begin{tabularx}{\linewidth}{@{}l X@{}}
Dec 2022 & Interview Training \\
Jul 2020 & Writing Workshop for Climate Scientists \\
Mar 2020 & \href{https://ers-imaging.github.io/uk_workshop}{ESA JWST Master Class} \\
Jul 2019 & \href{https://indico.ictp.it/event/8669}{ICTP Summer School on Convective Organization and Climate Sensitivity} \\
Apr 2018 & \href{https://www.nag.com/content/fortran-modernization-workshop}{Fortran Modernisation Workshop} \\
Jan 2018 & \href{http://www.petans.co.uk/courses/survival/huet-caebs/}{Helicopter Underwater Escape Training Course (CA-EBS)} \\
Dec 2017 & Sea Survival Course \\
Jun 2017 & Weather presenting \\
Feb 2017 & Level 1 First Aid for Field Work Course \\
Jan 2017 & Raspberry Pi Basics \\
Apr 2016 & WWRP/WCRP/Bolin Center Polar Prediction School \\
Dec 2014 & UK Met Office Unified Model Training \\
% Dec 2011 & \href{https://doi.org/10.1175/BAMS-D-11-00048.1}{Global Climate Change course} \\
\end{tabularx}

\section{Vocational Experience}
\begin{tabularx}{\linewidth}{@{}l X@{}}
Apr--Jun 2018 & Data Technician \\ & \href{https://github.com/IGPResearch}{Processing of meteorological data collected in the IGP field campaign} | University of East Anglia \\
2015--2018 & Founder and Leader \\ & \href{https://ueapy.github.io/}{Local Python Users Group} | University of East Anglia \\
Feb--Mar 2018 & Member of the Meteorology Team \\ & \href{https://twitter.com/igpresearch}{The Iceland-Greenland Seas Project (IGP) field campaign} | Akureyri, Iceland \\
% Characterising the atmospheric forcing and the ocean response of coupled atmosphere-ocean processes; in particular cold-air outbreaks in the vicinity of the marginal ice zone and their effect on the ocean
Mar 2015 & Rapporteur \\ & \href{https://highlatdynamics.w.uib.no}{Dynamics of Atmosphere-Ice-Ocean Interactions in the High-Latitudes} | Rosendal, Norway \\
Aug--Sep 2013 & Weather Forecaster \\ & Forecast and Briefing Service | Main Aviation Meteorological Centre, Vnukovo Airport \\
% Aug 2012 & Field practice in meteorology | Study of prevailing mesoscale processes via wind characteristic measurements and lake hydrothermodynamical modelling | Kamchatka, Russia \\
% Jan--Feb 2012 & Field practice in meteorology | Karel Republic, Russia \\
\end{tabularx}

%----------------------------------------------------------------------------------------
% Impact and outreach
%----------------------------------------------------------------------------------------
\section{Impact}
\begin{itemize}[nosep, leftmargin=10pt]
    \item Press Releases
    \begin{itemize}
        \item Joint PR from the \href{https://www.exeter.ac.uk/research/news/articles/questtouncoverintricacies.html}{University of Exeter}, \href{https://www.american.edu/news/20220811-trappist-1.cfm}{American University}, \& \href{https://www.insu.cnrs.fr/fr/cnrsinfo/la-quete-pour-percer-le-mystere-des-climats-de-planetes-similaires-la-terre-avance}{INSU CNRS} on the THAI project.
    \end{itemize}
    \item Interviews
    \begin{itemize}
        \item Featured in the \href{https://theworld.org/stories/2018-08-03/sea-and-sky-scientists-brave-wicked-weather-explore-key-ocean-current}{PRI podcast} on the IGP campaign
    \end{itemize}
    \item Visualisation
    \begin{itemize}
        \item \href{https://exetersciencecentre.org/gallery/denis_sergeev_uoe_exoplanets/}{``Exoplanetary Atmospheres''} at Science as Art Gallery (Exeter Science Centre)
        \item \href{https://phys.org/news/2020-06-presence-airborne-signify-habitability-distant.html}{3D visualisation of dusty atmospheres for a press release}
    \end{itemize}
    \item School Visits
    \begin{itemize}
        \item Visit to Pool Academy as part of the \href{https://physics-astronomy.exeter.ac.uk/exoplanetexplorers/}{``Exoplanet Explorers'' programme}
    \end{itemize}
    \item Scientific Consulting
    \begin{itemize}
        \item \href{https://www.wethecurious.org/curious-stuff/stargazing-night-sky/exoplanet-explorers}{Videogame ``Exoplanet Explorers''} (STFC Nucleus grant ``4EP'', PI Prof. N. J. Mayne)
    \end{itemize}
    \item Blogging
    \begin{itemize}
        \item \href{http://www.scisnack.com/2015/12/17/disastrous-disaster-movies}{Disastrous Disaster Movies}
        \item \href{http://www.scisnack.com/2015/03/04/polar-lows-what-fuels-arctic-hurricanes}{Polar Lows: What Fuels Arctic Hurricanes?}
        \item \href{http://www.scisnack.com/2014/12/17/worldwide-weird-weather-words}{Worldwide Weird Weather Words}
    \end{itemize}
    % \item Miscellaneous
    % \begin{itemize}
    %     \item Featured article in QJRMS
    % \end{itemize}
\end{itemize}

%----------------------------------------------------------------------------------------
% Service
%----------------------------------------------------------------------------------------
\section{Academic Community}
\begin{itemize}[nosep, leftmargin=10pt]
    \item Organisation of Scientific Meetings
    \begin{itemize}
        \item ExoSLAM Summer School (Co-chair) | Exeter, Jun 2023 | $\sim$50 attendees
        \item Exoclimes VI (Member of LOC) | Exeter, Jun 2023 | $\sim$200 attendees
        \item Exeter Exoplanet Theory Group Summer Retreat | Mawgan Porth, Aug--Sep 2022 | 15 attendees
    \end{itemize}
    \item Committees and steering groups
    \begin{itemize}
        \item Climates Using Interactive Suites of Intercomparisons Nested for Exoplanet Studies (CUISINES)
    \end{itemize}
    \item Reviewing
    \begin{itemize}
        \item Journals: Geophysical Research Letters, Nature Astronomy, Astrophysical Journal, Planetary and Space Studies, Quarterly Journal of the Royal Meteorological Society
        \item Grants: STFC (Consolidated)
    \end{itemize}
    \item Professional Memberships: Fellow of the Royal Astronomical Society, Member of the Europlanet Society
\end{itemize}

\end{document}
